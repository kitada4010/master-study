\documentclass{jsarticle}
\usepackage{amsmath}
\usepackage[dvipdfmx]{graphicx}
\usepackage{mathrsfs}
\makeatletter

\def\@thesis{機械学習特論}
\def\id#1{\def\@id{#1}}
\def\department#1{\def\@department{#1}}

\def\@maketitle{
\begin{center}
{\huge \@thesis \par} %修士論文と記載される部分
\vspace{10mm}
{\LARGE\bf \@title \par}% 論文のタイトル部分
\vspace{10mm}
{\Large \@date\par}	% 提出年月日部分
\vspace{20mm}
{\Large \@department \par}	% 所属部分
{\Large 学籍番号 \@id \par}	% 学籍番号部分
\vspace{10mm}
{\large \@author}% 氏名 
\end{center}
\par\vskip 1.5em
}

\makeatother

\title{レポート課題 1}
\date{\today}
\department{創成科学研究科 基盤科学系 情報科学コース}
\id{19-8801-009-1}
\author{北田 和}
\begin{document}
\maketitle
\newpage


\section{}

\subsection{目的}
最小二乗法の精度がどの程度のものか検討する. 

\subsection{手法}
$M$ 個のデータをランダムに生成し, 特徴量$x_i(0, 1, \dots, M)$のデータとした(ランダムさは, 正規分布に従い, これ以降使用するランダムな数字も正規分布に従う). 
これに対し, 観測データ$y_i(0, 1, \dots, M)$を式(\ref{ymake})によって作成する.
\begin{equation}
\label{ymake}
y_i = \beta \times x_i + \beta_i \times 0.5
\end{equation}
ここで, $\beta$はランダムな定数であり, $i$に関係なく一つに決める. 
一方, $\beta_i(0, 1, \dots, M)$は, $i$によって異なるランダムな数である. 
また, $x_i$に関係ないこの項はノイズに該当する. 
なお, ランダムな数は, 正規分布に従うため, 平均は0となる. 
よって, 真の関数は, 式(\ref{y}) で表すことができる. 
\begin{equation}
\label{y}
y = \beta \times x
\end{equation}
作成した特徴量$x_i$と観測データ$y_i$から, 最小二乗法によって式(\ref{y})の$\beta$を, 予測する. 
ここで, 予測した傾きを$\beta^*$とすると, 最小二乗法によって求めた, 式(\ref{y})は式(\ref{ypred})のように表せる. 
\begin{equation}
\label{ypred}
y = \beta^* \times x 
\end{equation}


\subsection{結果}
生成した$\beta$は   , 
最小二乗法によって求めた$\beta *$は    ,であった. 



ここで, 観測データと式(\ref{y}), 式(\ref{ypred})を図\ref{kaku}にまとめた. 横軸は特徴量 $x$ , 縦軸は観測データ$y$である. 


\subsection{考察}






\section{}
\subsection{目的}
リッジ回帰における正則化パラメータ$\alpha$と予測誤差の関係を調べる. 

\subsection{手法}
python のライブラリである scikit-learn 内に存在する boston housing データ使用した. 
boston housing データには, リッジ回帰





\end{document}
