\documentclass{beamer}
\usepackage{bm}
\usepackage{amsmath}
%\ usetheme{Boadilla}
\begin{document}


\begin{frame}{3.3(5) $x = 1$ (2次元)}
  \begin{eqnarray*}
    x &=& 1\\
    x-1 &=& 0
  \end{eqnarray*}

  ここで, 3.1(2)より
  \begin{eqnarray*}
    \frac{x-a}{u} &=& \frac{y-b}{v} \\
    v(x-b) &=& u(y-b)
  \end{eqnarray*}
  であるから, 
  \begin{eqnarray*}
    \bm{p} &=& 
    \begin{pmatrix}
      a\\
      b
    \end{pmatrix} =
    \begin{pmatrix}
      1\\
      0
    \end{pmatrix} \\
    \bm{u} &=&
    \begin{pmatrix}
      u\\
      v
    \end{pmatrix} =
    \begin{pmatrix}
      0\\
      1
    \end{pmatrix}
  \end{eqnarray*}
  よって, (1, 0)を通り, (0, 1)に平行な直線
\end{frame}


\begin{frame}{3.3(6) $x = 1$ (3次元)}
  \begin{eqnarray*}
    x &=& 1\\
    x-1 &=& 0
  \end{eqnarray*}

  ここで, 3.2(2)より
  \begin{equation*}
    a(x-x_0) + b(y-y_0) + c(z-z_0) = 0
  \end{equation*}
  であるから, 
  \begin{equation*}
    \bm{p} = 
    \begin{pmatrix}
      x_0\\
      y_0\\
      z_0
    \end{pmatrix} =
    \begin{pmatrix}
      1\\
      0\\
      0
    \end{pmatrix} , 
    \bm{t} = 
    \begin{pmatrix}
      a\\
      b\\
      c
    \end{pmatrix} =
    \begin{pmatrix}
      1\\
      0\\
      0
    \end{pmatrix}
  \end{equation*}
  よって, (1, 0, 0)を通り, (1, 0, 0)に直交する平面
\end{frame}



\begin{frame}{3.3(7) $x = 1, y=0$ (3次元)}
  \begin{equation*}
    \begin{array}{ccc}
      x &=& 1\\
      x-1 &=& 0
    \end{array},
    \begin{array}{ccc}
      y &=& 0\\
      &&
    \end{array}
  \end{equation*}

  ここで, 3.1(3)より
  \begin{eqnarray*}
    \frac{x-a}{u} &=& \frac{z-c}{w} \\
    w(x-a) &=& u(z-c) \\
    \frac{y-b}{v} &=& \frac{z-c}{w} \\
    w(y-b) &=& v(z-c)
  \end{eqnarray*}
  であるから, 
  \begin{equation*}
    \bm{p} = 
    \begin{pmatrix}
      a\\
      b\\
      c
    \end{pmatrix} =
    \begin{pmatrix}
      1\\
      0\\
      0
    \end{pmatrix} , 
    \bm{u} =
    \begin{pmatrix}
      u\\
      v\\
      w
    \end{pmatrix} = 
    \begin{pmatrix}
      0\\
      0\\
      1
    \end{pmatrix}
  \end{equation*}
  よって, (1, 0, 0)を通り, (0, 0, 1)に平行な直線
\end{frame}



\begin{frame}{3.3(8) $|x| + |y| + |z|  = 1$}
  以下の8つの場合にわけられる
  \begin{eqnarray*}
    x > 0, y > 0, z > 0 \\
    x < 0, y > 0, z > 0 \\
    x > 0, y < 0, z > 0 \\
    x > 0, y > 0, z < 0 \\
    x < 0, y < 0, z > 0 \\
    x < 0, y > 0, z < 0 \\
    x > 0, y < 0, z < 0 \\
    x < 0, y < 0, z < 0 
  \end{eqnarray*}
\end{frame}


\begin{frame}{3.3(8) $|x| + |y| + |z|  = 1$}
  $x, y, z$が入れ替わっただけであることから, \\
  計算的には4つ場合にわけられる
  \begin{eqnarray}
    \label{sei}
    && x > 0, y > 0, z > 0 \\
    \label{hu1}
    && \left\{
    \begin{array}{c}
    x < 0, y > 0, z > 0 \\
    x > 0, y < 0, z > 0 \\
    x > 0, y > 0, z < 0 
    \end{array} \right. \\
    \label{hu2}
    && \left\{
    \begin{array}{c}
    x < 0, y < 0, z > 0 \\
    x < 0, y > 0, z < 0 \\
    x > 0, y < 0, z < 0 
    \end{array} \right. \\
    \label{hu}
    && x < 0, y < 0, z < 0 
  \end{eqnarray}
\end{frame}




\begin{frame}{3.3(8) $|x| + |y| + |z|  = 1$ \hspace{5mm} (\ref{sei})全ての成分が正の場合}
  \begin{eqnarray*}
    x + y + z &=& 1\\
    (x-1) + y + z &=& 0\\
    x + (y-1) + z &=& 0\\
    x + y + (z-1) &=& 0
  \end{eqnarray*}
  ここで, 3.2(2)より
  \begin{equation*}
    a(x-x_0) + b(y-y_0) + c(z-z_0) = 0
  \end{equation*}
  であるから,
  \begin{equation*}
    \bm{p} = 
    \begin{pmatrix}
      x_0\\
      y_0\\
      z_0
    \end{pmatrix} =
    \begin{pmatrix}
      1\\
      0\\
      0
    \end{pmatrix} , 
    \bm{t} = 
    \begin{pmatrix}
      a\\
      b\\
      c
    \end{pmatrix} =
    \begin{pmatrix}
      1\\
      1\\
      1
    \end{pmatrix}
  \end{equation*}
  よって, (1, 0, 0)を通り, (1, 1, 1)に直交する平面
\end{frame}

\begin{frame}{3.3(8) $|x| + |y| + |z|  = 1$ \hspace{5mm} (\ref{sei})全ての成分が正の場合}
  \begin{eqnarray*}
    x + y + z &=& 1\\
    (x-1) + y + z &=& 0\\
    x + (y-1) + z &=& 0\\
    x + y + (z-1) &=& 0
  \end{eqnarray*}
  ここで, 3.2(2)より
  \begin{equation*}
    a(x-x_0) + b(y-y_0) + c(z-z_0) = 0
  \end{equation*}
  であるから,
  \begin{equation*}
    \bm{p} = 
    \begin{pmatrix}
      x_0\\
      y_0\\
      z_0
    \end{pmatrix} =
    \begin{pmatrix}
      1\\
      0\\
      0
    \end{pmatrix} , 
    \begin{pmatrix}
      0\\
      1\\
      0
    \end{pmatrix} , 
    \begin{pmatrix}
      0\\
      0\\
      1
    \end{pmatrix} 
  \end{equation*}
  よって, (1, 0, 0), (0, 1, 0), (0, 0, 1) を通る平面
\end{frame}




\begin{frame}{3.3(8) $|x| + |y| + |z|  = 1$ \hspace{5mm} (\ref{hu1})1つの成分が負の場合}
  \begin{eqnarray*}
    -x + y + z &=& 1\\
    -(x+1) + y + z &=& 0\\
    x + (y-1) + z &=& 0\\
    x + y + (z-1) &=& 0
  \end{eqnarray*}
  ここで, 3.2(2)より
  \begin{equation*}
    a(x-x_0) + b(y-y_0) + c(z-z_0) = 0
  \end{equation*}
  であるから,
  \begin{equation*}
    \bm{p} = 
    \begin{pmatrix}
      x_0\\
      y_0\\
      z_0
    \end{pmatrix} =
    \begin{pmatrix}
      -1\\
      0\\
      0
    \end{pmatrix} , 
    \begin{pmatrix}
      0\\
      1\\
      0
    \end{pmatrix} , 
    \begin{pmatrix}
      0\\
      0\\
      1
    \end{pmatrix} 
  \end{equation*}
  よって, (-1, 0, 0), (0, 1, 0), (0, 0, 1) を通る平面
\end{frame}



\begin{frame}{3.3(8) $|x| + |y| + |z|  = 1$ \hspace{5mm} (\ref{hu1})1つの成分が負の場合}
  同様に, $x > 0, y < 0, z > 0$ の場合 
  \begin{eqnarray*}
    x - y + z &=& 1\\
    (x-1) - y + z &=& 0\\
    x - (y+1) + z &=& 0\\
    x - y + (z-1) &=& 0
  \end{eqnarray*}
  (1, 0, 0), (0, -1, 0), (0, 0, 1) を通る平面 \\ \vspace{5mm}

  $x > 0, y > 0, z < 0$ の場合 
  \begin{eqnarray*}
    x - y + z &=& 1\\
    (x-1) + y - z &=& 0\\
    x + (y-1) - z &=& 0\\
    x + y - (z+1) &=& 0
  \end{eqnarray*}
  (1, 0, 0), (0, 1, 0), (0, 0, -1) を通る平面
\end{frame}


\begin{frame}{3.3(8) $|x| + |y| + |z|  = 1$ \hspace{5mm} (\ref{hu2})2つの成分が負の場合}
  \begin{eqnarray*}
    -x - y + z &=& 1\\
    -(x+1) - y + z &=& 0\\
    -x - (y+1) + z &=& 0\\
    -x - y + (z-1) &=& 0
  \end{eqnarray*}
  ここで, 3.2(2)より
  \begin{equation*}
    a(x-x_0) + b(y-y_0) + c(z-z_0) = 0
  \end{equation*}
  であるから,
  \begin{equation*}
    \bm{p} = 
    \begin{pmatrix}
      x_0\\
      y_0\\
      z_0
    \end{pmatrix} =
    \begin{pmatrix}
      -1\\
      0\\
      0
    \end{pmatrix} , 
    \begin{pmatrix}
      0\\
      -1\\
      0
    \end{pmatrix} , 
    \begin{pmatrix}
      0\\
      0\\
      1
    \end{pmatrix} 
  \end{equation*}
  よって, (-1, 0, 0), (0, -1, 0), (0, 0, 1) を通る平面
\end{frame}



\begin{frame}{3.3(8) $|x| + |y| + |z|  = 1$ \hspace{5mm} (\ref{hu2})2つの成分が負の場合}
  同様に, $x < 0, y > 0, z < 0$ の場合 
  \begin{eqnarray*}
    -x + y -z &=& 1\\
    -(x+1) + y - z &=& 0\\
    -x + (y-1) - z &=& 0\\
    -x + y - (z+1) &=& 0
  \end{eqnarray*}
  (-1, 0, 0), (0, 1, 0), (0, 0, -1) を通る平面 \\ \vspace{5mm}

  $x > 0, y < 0, z < 0$ の場合 
  \begin{eqnarray*}
    x - y - z &=& 1\\
    (x-1) - y - z &=& 0\\
    x - (y+1) - z &=& 0\\
    x - y - (z+1) &=& 0
  \end{eqnarray*}
  (1, 0, 0), (0, -1, 0), (0, 0, -1) を通る平面
\end{frame}



\begin{frame}{3.3(8) $|x| + |y| + |z|  = 1$ \hspace{5mm} (\ref{hu})全ての成分が負の場合}
  \begin{eqnarray*}
    -x - y - z &=& 1\\
    -(x+1) - y - z &=& 0\\
    -x - (y+1) - z &=& 0\\
    -x - y - (z+1) &=& 0
  \end{eqnarray*}
  ここで, 3.2(2)より
  \begin{equation*}
    a(x-x_0) + b(y-y_0) + c(z-z_0) = 0
  \end{equation*}
  であるから,
  \begin{equation*}
    \bm{p} = 
    \begin{pmatrix}
      x_0\\
      y_0\\
      z_0
    \end{pmatrix} =
    \begin{pmatrix}
      -1\\
      0\\
      0
    \end{pmatrix} , 
    \begin{pmatrix}
      0\\
      -1\\
      0
    \end{pmatrix} , 
    \begin{pmatrix}
      0\\
      0\\
      -1
    \end{pmatrix} 
  \end{equation*}
  よって, (-1, 0, 0), (0, -1, 0), (0, 0, -1) を通る平面
\end{frame}

\begin{frame}{3.3(8) $|x| + |y| + |z|  = 1$ \hspace{5mm} まとめ}
  $x > 0, y > 0, z > 0$ の場合 \\
  (1, 0, 0), (0, 1, 0), (0, 0, 1) を通る平面
  
  $x < 0, y > 0, z > 0$ の場合 \\
  (-1, 0, 0), (0, 1, 0), (0, 0, 1) を通る平面
  
  $x > 0, y < 0, z > 0$ の場合 \\
  (1, 0, 0), (0, -1, 0), (0, 0, 1) を通る平面

  $x > 0, y > 0, z < 0$ の場合 \\
  (1, 0, 0), (0, 1, 0), (0, 0, -1) を通る平面

  $x < 0, y < 0, z > 0$ の場合 \\
  (-1, 0, 0), (0, -1, 0), (0, 0, 1) を通る平面
  
  $x < 0, y > 0, z < 0$ の場合 \\
  (-1, 0, 0), (0, 1, 0), (0, 0, -1) を通る平面

  $x > 0, y < 0, z < 0$ の場合 \\
  (1, 0, 0), (0, -1, 0), (0, 0, -1) を通る平面
  
  $x < 0, y < 0, z < 0$ の場合 \\
  (-1, 0, 0), (0, -1, 0), (0, 0, -1) を通る平面
\end{frame}

\end{document}

