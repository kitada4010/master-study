\documentclass[12pt]{jreport}
\usepackage[top=30truemm,bottom=30truemm,left=25truemm,right=25truemm]{geometry}
\usepackage{amsmath}
\setcounter{secnumdepth}{5}
\usepackage[dvipdfmx]{graphicx}
%\usepackage{verbatim}
%\usepackage[dvipdfmx]{color}
%\usepackage{bmpsize}
%\usepackage[dvipdfmx]{graphicx}
%\usepackage[dvipdfmx]{graphicx}
%\usepackage{graphicx}
%\usepackage[dviout]{graphicx}
\usepackage{braket}
\usepackage{verbatim}
\usepackage{cite}
\usepackage{caption}
%\usepackage[subrefformat=parens]{subcaption}
%\usepackage[hang,small,bf]{caption}
\usepackage{subcaption}
\captionsetup{compatibility=false}
\graphicspath{{images/}}
%\graphicspath{{images/presen/}}
\begin{document}


\chapter{解析手法}
海馬で観測された神経の発火情報に対して, 0.4 ms のスパイク頻度を計算しスパイク頻度データであるPSTH(Peri-Stimulus Time Histogram)を作成した.
この作成した PSTH データにおけるパターンの候補として考えられる数は,
(最大スパイク頻度)の(調べるパターン長)乗である.
このパターンの候補が, 神経経験前, 後のPSTHデータにおいて, 出現した回数を求め, それぞれ, 出現確率を計算した.
ここで, パターン $i$ の経験前出現確率を $P_b(i)$, 経験後出現確率を $P_a(i)$ として, どれだけ出現確率が変化しているかをカルバック・ライブラー情報量を用いて比較した. 
なお, カルバック・ライブラー情報量 $D_{KL}(P_b || P_a)$ は, 式(\ref{kullback}) のように計算できる.


\begin{equation}
  \label{kullback}
  D_{KL}(P_b || P_a) = \sum_i P_b(i) \log{P_b(i)}{P_a(i)}
\end{equation}


\chapter{結果}



\end{document}
% \chapter{新奇経験内容により脳活動に出現するリップルの変化}
% %\section{各エピソードごとのリップル変化を調べることで新奇経験内容とリップル波形の対応を見つける}
% \section{新奇経験内容とリップルの対応を見つける}
% 前の章で作成したプログラムを使用し, 新奇経験前後で変化するリップル抽出の特徴を調べることにした. 
% %新奇経験によるリップルの変化を見る
% %\subsection{リップル出現頻度と新奇経験が対応している可能性の検討}
% %\subsection{リップルが短期記憶を長期記憶に変換している可能性の検討}
% %\subsection{リップル波形と新奇経験内容が対応している可能性の検討}
% %\section{各エピソードと新奇経験後出現したリップルの関係}



% \section{リップル比較手法}
% % \subsection{リップル出現頻度変化}
% % 新奇経験前で 1 分あたりに出現したリップル数を 1 とした時に, 新奇経験中と新奇経験後でリップル出現数がいくつになったかを計算した. 
% % 加えて, 実験を 1 分ごとに区切った時のリップル出現数を求め方グラフ化し, それぞれ, 個体間, エピソード間で比較した. 

% \subsection{ウェーブレット変換によるリップル波形の周波数解析}
% 新奇経験前, 中, 後で抽出できたリップルに対し, ウェーブレット変換を使用して, リップル波形の周波数特性に変化が見られるか検討した. 

% \subsection{{\sl SVM}の識別関数の係数に注目したリップル波形の周波数解析}
% % 新奇経験前後のリップルに対し, それぞれラベル付けを行い, それぞれ{\sl SVM}を用いて学習を行わせた. 
% % その後, それぞれの{\sl SVM}の識別関数の係数をグラフ化し比較し, 周波数特性に変化が見られるか検討した. 
% % 加えて, 新奇経験前の個体間や, 新奇経験によって変化した周波数特性に傾向がないかを個体間やエピソード間で比較した. 
% 拘束ストレスを与えた




%\section{リップル比較の結果}
%\section{新奇経験前と経験後のリップル比較}
%\subsection{新奇経験前と経験後のリップル比較結果}
%\subsubsection{各エピソードごとのリップル出現頻度の比較}
%\subsubsection{各エピソードごとのリップル周波数特性の比較}
%\subsection{新規経験前と経験後のリップルを比較した考察}

%\section{新規経験前のリップル比較}
%\subsection{新規経験前のリップル比較結果}
%\subsubsection{各エピソード間での新規経験前リップル出現頻度の比較}
%\subsubsection{各エピソード間での新規経験前リップル周波数特性の比較}
%\subsection{新規経験前のリップル比較した考察}

% \section{新奇経験後のリップルを各エピソードで比較}
% \subsection{新奇経験後のリップルを各エピソードで比較結果}
% \subsubsection{新規経験後のリップル出現頻度変化を各エピソードで比較}
% \subsubsection{新規経験後のリップル周波数特性変化を各エピソードで比較}
% \subsection{新奇経験後のリップル変化を各エピソードで比較した考察}


%\section{を用いた各エピソードごとに新奇経験後に出現したリップルの周波数特性変化の比較}
%\subsection{object に出現したリップルの周波数特性}
%\subsection{maleに出現したリップルの周波数特性}
%\subsection{femaleに出現したリップルの周波数特性}
%\subsection{restraintに出現したリップルの周波数特性}
%\subsection{各エピソードにおけるリップル周波数特性の比較結果と考察}

%\section{比較した全体を通しての考察}
%\section{考察}







\chapter{まとめ}
神経活動におけるリップル波形の特徴解析をするために, 覚醒時のリップル波形に注目し, リップル波形と, ノイズ波形, ベース波形の周波数分布の違いを調べた. 
周波数解析方法として FFT とウェーブレット変換を使用し, リップル波形のパワースペクトログラムを作成し解析した. 
%ところ, 
リップル波形については, FFT によるパワースペクトログラムの場合, リップル波形の出現時に全帯域において周波数強度が強いことが確認できた. 
よって, 先行研究と類似した 180 Hz から 200 Hz の特徴が確認できた. 
一方, ウェーブレット変換によるパワースペクトログラムの場合, 180 Hz から 200 Hz の周波数強度よりも, 700 Hz 以上の周波数強度の特徴が強いことが確認できた. 
加えて, 600 Hz 以下のの周波数強度分布はベース波形でも確認できたため, 600 Hz 以上の周波数強度分布がリップル波形で特徴的な活動である. 
%よって, リップル波形の時間変化を詳しく見るためには, ウェーブレット変換によるパワースペクトログラムの方が良かった. 
%が, 海馬の神経活動からリップルを自動判定する上では, 
%FFT の方が, 計算量面でも, 精度面でも良かった. 


%しかし, FFT とウェーブレット変換の周波数強度データを$k$近傍法と{\sl SVM} によって, 自動判定ができるか学習させたところ識別率は FFT のほうがよかった. 
リップル波形の自動抽出を検討するために, 
$k$ 近傍法による自動判定ができるか試したところ, FFT による周波数強度データを用いた場合は 90 \% 前後, 
ウェーブレット変換による周波数強度データを用いた場合は 80 \% 前後と高い識別率が得られた. 
一方, {\sl SVM}による自動判定では, FFT による周波数強度データを用いた場合は 80 \% 前後, 
ウェーブレット変換による周波数強度データを用いた場合は 55 \% 前後と $k$ 近傍法によって得られた結果より識別率は低かった. 
そこで, 精度の良かった FFT 周波数強度データを規格化し, {\sl SVM}の超平面の係数を確認したところ, 
リップル波形で重要な周波数成分は 500 Hz 以上の高周波成分であることがわかった. 
%リップル波形の FFTによる周波数強度 とそれ以外の波形は高周波の特徴が異なった. しかし, リップル波形のウェーブレット変換はリップル波形開始終了時刻の前後ではあまりリップル波形の特性が見られないためか, 精度が悪かった. 
また, この自動判定システムによるリップル波形の自動抽出精度を検討したところ, 
新しく用意した特徴量データに対しての識別率は低かった. この結果は, 時間変化や新奇経験内容によって神経活動にも変化があり, リップル波形を正しく分類できなくなる事を示唆させた. 
また, 学習に利用した特徴量である周波数強度は, 高周波の時間分解能が低いため, 今後, 時間窓を変更した FFT によって求めた周波数強度を特徴量として利用した場合の結果はどうなるか等を検討する必要がある. 



%今後の展望として, データ数, 個体数を増やす, FFT の窓をより短くする, ウェーブレット変換による判定方法の改良を考えるなどを考えている. 

%ラットに, オス, メス, 新奇物体と遭遇する経験と拘束される経験の4種類の新奇経験をラットに与えた. 
%すると, オスと遭遇した時はリップル波形は???のように変化し, 
%メスと遭遇した時はリップル波形は???のように変化し, 
%新奇物体と遭遇した時はリップル波形は???のように変化し, 
%拘束された時はリップル波形は???のように変化した. 
% また, 個体間では???は同一であったが, ???が変化した. 
% 全てのリップルで, ???は同一であるが, 新奇経験内容で???が変化した. 


% \begin{table}
%   \scalebox{0.4}[0.4]{
%     \input{./tabledata/B39Rrestraint/knn150-B39Rrestraint-pre-post.tex}
%   }
% \end{table}

新奇経験によってリップル波形に変化が見られるか検討したところ, リップル, ベース波形ともに 90 \% 前後と新奇経験前後の識別率が高かった. 
また, 誤判定においても, 新奇経験前後を誤った場合は, 0.5 \% 前後であり, 殆どは, 新奇経験前のリップル波形を新奇経験前のリップル・ベース波形と判定したり, 新奇経験前のリップル・ベース波形をリップル波形や, ベース波形と判定するものであったため, 大きな問題はない. 
よって, 新奇経験によってベース波形を含む神経活動が変化したことが示唆された. 







\chapter*{謝辞}
\addcontentsline{toc}{chapter}{謝辞}
本研究を行うにあたって, ラットの海馬神経活動データと多大な助言を山口大学大学院医学系研究科の石川淳子助教にご提供頂きました. 心よりお礼申し上げます. 本卒業論文中のデータ解析をする上で, 西井淳教授には丁寧なご指導を頂き深く感謝しております. また, 共同研究をさせて頂く機会をくださった山口大学大学院医学系研究科の美津島大教授に厚く感謝を申し上げます. 
加えて, 卒論を書く上で相談に乗っていただいた生体情報システム研究室の皆様には大変お世話になりました. 
最後になりますが, ここまで公私共にサポートしてくださった皆様, 家族に感謝致します. 


%\chapter*{参考文献}
\addcontentsline{toc}{chapter}{参考文献}
\bibliography{myrefs} 
\bibliographystyle{unsrt}


\end{document}
